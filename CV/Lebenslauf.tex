%-------------------------
% Resume in Latex
% Author : Jake Gutierrez
% Based off of: https://github.com/sb2nov/resume
% License : MIT
%------------------------

\documentclass[letterpaper,11pt]{article}

\usepackage{latexsym}
\usepackage[empty]{fullpage}
\usepackage{titlesec}
\usepackage{marvosym}
\usepackage[usenames,dvipsnames]{color}
\usepackage{verbatim}
\usepackage{enumitem}
\usepackage[hidelinks]{hyperref}
\usepackage{fancyhdr}
\usepackage[english]{babel}
\usepackage{tabularx}
\usepackage{fontawesome}
\input{glyphtounicode}


%----------FONT OPTIONS----------
% sans-serif
% \usepackage[sfdefault]{FiraSans}
% \usepackage[sfdefault]{roboto}
% \usepackage[sfdefault]{noto-sans}
% \usepackage[default]{sourcesanspro}

% serif
% \usepackage{CormorantGaramond}
% \usepackage{charter}


\pagestyle{fancy}
\fancyhf{} % clear all header and footer fields
\fancyfoot{}
\renewcommand{\headrulewidth}{0pt}
\renewcommand{\footrulewidth}{0pt}

% Adjust margins
\addtolength{\oddsidemargin}{-0.5in}
\addtolength{\evensidemargin}{-0.5in}
\addtolength{\textwidth}{1in}
\addtolength{\topmargin}{-.5in}
\addtolength{\textheight}{1.0in}

\urlstyle{same}

\raggedbottom
\raggedright
\setlength{\tabcolsep}{0in}

% Sections formatting
\titleformat{\section}{
  \vspace{-4pt}\scshape\raggedright\large
}{}{0em}{}[\color{black}\titlerule \vspace{0pt}]

% Ensure that generate pdf is machine readable/ATS parsable
\pdfgentounicode=1

%-------------------------
% Custom commands
\newcommand{\resumeItem}[1]{
  \item\small{
    {#1 \vspace{-2pt}}
  }
}

\newcommand{\resumeSubheading}[4]{
  \vspace{-2pt}\item
    \begin{tabular*}{0.97\textwidth}[t]{l@{\extracolsep{\fill}}r}
      \textbf{#1} & #2 \\
      \textit{\small#3} & \textit{\small #4} \\
    \end{tabular*}\vspace{-7pt}
}

\newcommand{\resumeSubSubheading}[2]{
    \item
    \begin{tabular*}{0.97\textwidth}{l@{\extracolsep{\fill}}r}
      \textit{\small#1} & \textit{\small #2} \\
    \end{tabular*}\vspace{-7pt}
}

\newcommand{\resumeProjectHeading}[2]{
    \item
    \begin{tabular*}{0.97\textwidth}{l@{\extracolsep{\fill}}r}
      \small#1 & #2 \\
    \end{tabular*}\vspace{-7pt}
}

\newcommand{\resumeSubItem}[1]{\resumeItem{#1}\vspace{-4pt}}

\renewcommand\labelitemii{$\vcenter{\hbox{\tiny$\bullet$}}$}

\newcommand{\resumeSubHeadingListStart}{\begin{itemize}[leftmargin=0.15in, label={}]}
\newcommand{\resumeSubHeadingListEnd}{\end{itemize}}
\newcommand{\resumeItemListStart}{\begin{itemize}}
\newcommand{\resumeItemListEnd}{\end{itemize}\vspace{-5pt}}

%-------------------------------------------
%%%%%%  RESUME STARTS HERE  %%%%%%%%%%%%%%%%%%%%%%%%%%%%

\begin{document}

%----------HEADING----------
% \begin{tabular*}{\textwidth}{l@{\extracolsep{\fill}}r}
%   \textbf{\href{http://sourabhbajaj.com/}{\Large Sourabh Bajaj}} & Email : \href{mailto:sourabh@sourabhbajaj.com}{sourabh@sourabhbajaj.com}\\
%   \href{http://sourabhbajaj.com/}{http://www.sourabhbajaj.com} & Mobile : +1-123-456-7890 \\
% \end{tabular*}

\begin{center}
  \textbf{\Huge \scshape Nikolai Madlener} \\ \vspace{1pt}
  % \small Bothmerstr. 19, 80634 Munich\\ 
  \href{https://www.linkedin.com/in/nikolai-madlener-163b14169/}{\faLinkedinSquare} $|$
  \small +43 680 2216544 $|$ {{nikolai.madlener@tum.de}} $|$
  \href{https://github.com/NikolaiMadlener}{\faGithub}
  
\end{center}

%-----------EDUCATION-----------
\section{Ausbildung}  
\resumeSubHeadingListStart
  \resumeSubheading
      {Technische Universität München}{Oktober 2021 -- Juli 2024 (voraussichtlicher Abschluss)}
      {Master of Science in Informatik}{München, Deutschland}
      \resumeItemListStart
        \resumeItem{Fokus auf Software Engineering, Machine Learning und Distributed Systems}
      \resumeItemListEnd
    \resumeSubheading
      {Peking University}{Februar 2023 -- Juli 2023}
      {General Visiting Student (Auslandssemester)}{Beijing, China}
      \resumeItemListStart
        \resumeItem{\href{https://newsen.pku.edu.cn/news_events/news/campus/13276.html}{PKU's erstes Exchange Cafe} über LLM's und ChatGPT mitorganisiert und als Sprecher aufgetreten}
      \resumeItemListEnd

    \resumeSubheading
      {Technische Universität München}{Oktober 2017 -- April 2021}
      {Bachelor of Science in Informatik}{München, Deutschland}
    \resumeSubheading
      {Höhere Technische Lehranstalt Rankweil}{September 2011 -- Juni 2016}
      {Fachbereich Bautechnik mit Vertiefung Heizung-Lüftung}{Rankweil, Österreich}
      \resumeItemListStart
      \resumeItem{Mit Auszeichnung abgeschlossen}
      \resumeItemListEnd
  \resumeSubHeadingListEnd

%-----------EXPERIENCE-----------
\section{Berufliche Erfahrung \& Praktikas}
  \resumeSubHeadingListStart

    \resumeSubheading
      {Software-Engineer (Teilzeit)}{Juli 2019 -- März 2023}
      {ALEAS AG}{Ruggel, Liechtenstein}
      \resumeItemListStart
        \resumeItem{Teilzeit Werkstudenten Tätigekeiten während des Studiums, v.A während der Semesterferien}
        \resumeItem{Full-Stack und App-Entwicklung mit ASP.NET und React Native}
      \resumeItemListEnd

    \resumeSubheading
      {Wissenschaftliche Hilfskraft}{Oktober 2021 -- März 2023}
      {Technische Universität München, Digital Health Group}{Munich, Germany}
      \resumeItemListStart
        \resumeItem{Entwicklung einer \href{https://github.com/dhg-applab/HeMo}{Navigations iOS App} die Nutzern hilft ihre Gesundheit und Fitness zu verbessern}
        % \resumeItem{Verbesserung der Performance der Rutensuche um mehr als 50\%}
      \resumeItemListEnd

    \resumeSubheading
      {Logistik Software System Konfiguration (Praktikum)}{März 2021 -- April 2021}
      {Tesla Inc.}{Grünheide, Deutschland}
      \resumeItemListStart
     
        \resumeItem{Erstkonfiguration des Materialflusssoftwaresystems für die Intralogistik im Werk Tesla Giga Berlin}
        \resumeItem{Requirements Engineering und Architekturkonzeption im Austausch mit Entwicklern und Materialflussplanern}
      \resumeItemListEnd

    % \resumeSubheading
    %   {iOS-Entwickler (Praktikum)}{Okt. 2019 -- Feb. 2020}
    %   {TUM/Carl Zeiss Meditec AG}{München, Deutschland}
    %   \resumeItemListStart
    %     \resumeItem{Enwicklung einer nativen iOS-App in einem agilen Team}
    %     \resumeItem{Im Zuge des Bachelor Praktikums (iPraktikum/Lehrstuhl für Angewandte Softwaretechnik)}
    %   \resumeItemListEnd

  \resumeSubHeadingListEnd

% %-----------PROJECTS-----------
% \section{Projekte \& Expertise}
%     \resumeSubHeadingListStart
%       \resumeProjectHeading
%           {\textbf{encore.} $|$ \emph{Swift, SwiftUI, Spotify API, REST, Git}}{Mai 2020 -- heute}
%           \resumeItemListStart
%             \resumeItem{Entwicklung einer nativen iOS-App mit SwiftUI für geteilte Echtzeit Musik-Warteschlangen}
%             \resumeItem{Integration von Spotify implementiert mithilfe der Spotify API}
%             \resumeItem{Anbindung an bestehenden Server mit der REST API}
%           \resumeItemListEnd
%       \resumeProjectHeading
%           {\textbf{Swift Tutor} }{April 2020}
%           \resumeItemListStart
%             \resumeItem{Als Tutor beim iPraktikum Bootcamp Studierende beim Einstieg in die iOS-Entwicklung unterstützt (TUM/Lehrstuhl für Angewandte Softwaretechnik)}
%             \resumeItem{Vermittelte Inhalte u. a. Swift, SwiftUI, UIKit, Server-Side Swift, UML}
%           \resumeItemListEnd

%     \resumeSubHeadingListEnd

%-----------EXTRACURRICULAR-----------
\section{Außercurriculare Tätigkeiten \& Ehrenamt}
\resumeSubHeadingListStart
\resumeProjectHeading
    {\textbf{MINGA Mentor} }{August 2022 -- Februar 2023}
    \resumeItemListStart
      \resumeItem{Mentor für Austauschstudierende um jene bei der Orientierung in München und an der TUM zu unterstützen}
    \resumeItemListEnd
\resumeProjectHeading
  {\textbf{Ferienakademie 2022 -- ”Decentralized Decision Making"}}{September 2022}
  \resumeItemListStart
    \resumeItem{Als ausgewählter Student bei der \href{https://ase.in.tum.de/lehrstuhl_1/projects/1193-ferienakademie-2022}{Ferienakademie} teilgenommen mit dem Ziel innovative Ideen zu erforschen}
    \resumeItem{Entwicklung eines selbst-organisiertem Roboter Lieferdienst mit Fokus auf dezentralisierte Entscheidungsfindung}
    % \resumeItem{Built an iOS app that viisualizes the robots' status/location in realtime using a message-ortiented architecture}
  \resumeItemListEnd
\resumeSubHeadingListEnd

%-----------AWARDS-----------
\section{Auszeichnungen}
\resumeSubHeadingListStart
\resumeProjectHeading
{\textbf{HackaTUM 2022 Gesamtsieger}}{November 2022}
  \resumeItemListStart
    \resumeItem{Gesamtsieger (aus über 200 Teams und mehr als 800 Teilnehmer) des \href{https://devpost.com/software/sixtcharge}{HackaTUM 2022}, einer der größten Hackathons Europas}
    % \resumeItem{Built an iOS app that viisualizes the robots' status/location in realtime using a message-ortiented architecture}
  \resumeItemListEnd
  \resumeProjectHeading
  {\textbf{Bosch Technik fürs Leben Preis}}{Juni 2016}
  \resumeItemListStart
    \resumeItem{Nominiert als eine der besten 5 Matura Arbeiten Österreichs im Bereich Gebäudetechnik}
    % \resumeItem{Built an iOS app that viisualizes the robots' status/location in realtime using a message-ortiented architecture}
  \resumeItemListEnd
\resumeSubHeadingListEnd

%-----------ZIVI-----------
\section{Zivildienst}
  \resumeSubHeadingListStart
    \resumeSubheading
      {Zivildiener als Rettungssanitäter}{September 2016 -- Mai 2017}
      {Österreichisches Rotes Kreuz}{Feldkirch, Österreich}
  \resumeSubHeadingListEnd

%-----------PROGRAMMING SKILLS-----------
\section{Kentnisse \& Interessen}
 \begin{itemize}[leftmargin=0.15in, label={}]
    \small{\item{
      \textbf{Fremdsprachen}{: Englisch (Sprachreferenzniveau B2), Chinesisch (HSK1)} \\
    %  \textbf{Developer Tools}{: Git, Docker, TravisCI, Google Cloud Platform, VS Code, Visual Studio, PyCharm, IntelliJ, Eclipse} \\
     \textbf{Interessen}{: Laufen, Wandern, Skitouren, Software-Entwicklung, maschinelles Lernen, iOS}
    }}
 \end{itemize}

%-------------------------------------------
\end{document}
